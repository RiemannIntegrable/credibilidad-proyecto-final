\section{Introducción}

\begin{frame}{Material de Referencia}
    \begin{block}{Fuentes Bibliográficas}
        Esta presentación está basada completamente en:

        \begin{itemize}
            \item \textbf{ACTEX Lesson 42}: Bayesian Estimation and Credibility - Continuous Prior \cite{actex_lesson_42}
            \begin{itemize}
                \item Teoría fundamental
                \item Ejercicios y aplicaciones
            \end{itemize}

            \vspace{0.5em}

            \item \textbf{ACTEX Lesson 35}: Bayesian Estimation and Credibility - Discrete Prior \cite{actex_35}
            \begin{itemize}
                \item Notación matemática adoptada
                \item Convenciones de subíndices
            \end{itemize}
        \end{itemize}
    \end{block}

    \begin{alertblock}{Nota}
        Todos los ejercicios, teoremas y definiciones provienen de estos manuales de estudio para el examen STAM.
    \end{alertblock}
\end{frame}

\begin{frame}{Motivación: El Problema del Asegurador}
    \begin{block}{Escenario}
        Un asegurador desea estimar la tasa de siniestralidad $\lambda$ de un nuevo asegurado, sabiendo que:
        \begin{itemize}
            \item El número de siniestros por año sigue una distribución Poisson($\lambda$)
            \item El parámetro $\lambda$ varía entre asegurados (heterogeneidad)
            \item Disponemos de información previa sobre la población
        \end{itemize}
    \end{block}

    \pause

    \begin{exampleblock}{Pregunta Fundamental}
        ¿Cómo combinar la información previa de la población con la experiencia observada del asegurado para obtener la mejor estimación de $\lambda$?
    \end{exampleblock}

    \pause

    \begin{alertblock}{Respuesta}
        \textbf{Métodos Bayesianos de Credibilidad} con distribuciones continuas
    \end{alertblock}
\end{frame}