% =============================================================================
% PAQUETES PARA NOTA T�CNICA EKG - SEGUROS DE VIDA INDIVIDUAL
% =============================================================================

% Configuración b�sica del documento
\usepackage[utf8]{inputenc}
\usepackage{fontspec}
% Para XeLaTeX con soporte de español
\usepackage[spanish]{babel}

% Geometría y layout
\usepackage{geometry}
\usepackage{fancyhdr}
\usepackage{lastpage}
\usepackage{multicol}
\usepackage{setspace}

% Manejo de colores e im�genes
\usepackage{xcolor}
\usepackage{graphicx}
\usepackage{eso-pic}  % Para colocar imágenes en posiciones absolutas
\usepackage{float}
\usepackage{wrapfig}

% Tablas avanzadas
\usepackage{booktabs}
\usepackage{longtable}
\usepackage{tabularx}
\usepackage{array}
\usepackage{multirow}
\usepackage{colortbl}

% Matem�ticas y notaci�n actuarial
\usepackage{amsmath}
\usepackage{amssymb}
\usepackage{amsfonts}
\usepackage{arev}  % Matemáticas sans serif tipo Vera Sans, muy legible

% Algoritmos (para pseudocódigo en credibilidad)
\usepackage{algorithm}
\usepackage{algorithmic}

% Renombrar "Algorithm" a "Algoritmo" en español
\floatname{algorithm}{Algoritmo}

% Traducir palabras clave de algorithmic al español
\renewcommand{\algorithmicrequire}{\textbf{Entrada:}}
\renewcommand{\algorithmicensure}{\textbf{Salida:}}
\renewcommand{\algorithmicend}{\textbf{fin}}
\renewcommand{\algorithmicif}{\textbf{si}}
\renewcommand{\algorithmicthen}{\textbf{entonces}}
\renewcommand{\algorithmicelse}{\textbf{si no}}
\renewcommand{\algorithmicelsif}{\textbf{si no, si}}
\renewcommand{\algorithmicendif}{\textbf{fin si}}
\renewcommand{\algorithmicfor}{\textbf{para}}
\renewcommand{\algorithmicforall}{\textbf{para todo}}
\renewcommand{\algorithmicdo}{\textbf{hacer}}
\renewcommand{\algorithmicendfor}{\textbf{fin para}}
\renewcommand{\algorithmicwhile}{\textbf{mientras}}
\renewcommand{\algorithmicendwhile}{\textbf{fin mientras}}
\renewcommand{\algorithmicloop}{\textbf{repetir}}
\renewcommand{\algorithmicendloop}{\textbf{fin repetir}}
\renewcommand{\algorithmicrepeat}{\textbf{repetir}}
\renewcommand{\algorithmicuntil}{\textbf{hasta}}
\renewcommand{\algorithmicprint}{\textbf{imprimir}}
\renewcommand{\algorithmicreturn}{\textbf{devolver}}
\renewcommand{\algorithmictrue}{\textbf{verdadero}}
\renewcommand{\algorithmicfalse}{\textbf{falso}}
\renewcommand{\algorithmiccomment}[1]{\hfill $\triangleright$ #1}

% Gr�ficos y diagramas
\usepackage{tikz}
\usepackage{pgfplots}
\pgfplotsset{compat=1.17}
\usetikzlibrary{arrows.meta, positioning, shapes, calc}

% Listas y enumeraciones
\usepackage{enumitem}

% Enlaces y referencias
\usepackage{hyperref}
\usepackage{url}

% Gesti�n de t�tulos y secciones
\usepackage{titlesec}
\usepackage{titletoc}

% Ap�ndices
\usepackage[toc,page]{appendix}

% Bibliograf�a
% Bibliografía - natbib (compatible con XeLaTeX y BibTeX)
\usepackage[numbers,sort&compress]{natbib}
% Opciones:
% - numbers: estilo numérico [1], [2], etc.
% - authoryear: estilo (Autor, año) - cambia a este si lo prefieres
% - sort&compress: ordena y comprime referencias múltiples [1-3] en lugar de [1,2,3]

% Otros paquetes �tiles
\usepackage{lipsum}
\usepackage{textcomp}
\usepackage{gensymb}
\usepackage{siunitx}

% Cajas y alertas
\usepackage{tcolorbox}

% Código fuente
\usepackage{listings}