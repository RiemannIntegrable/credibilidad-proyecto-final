% =============================================================================
% COMANDOS PERSONALIZADOS PARA NOTA TÉCNICA EKG (SIMPLIFICADO)
% =============================================================================

% -----------------------------------------------------------------------------
% COMANDOS DE BIBLIOGRAFÍA
% -----------------------------------------------------------------------------
% Los comandos de citación (\cite, \citep, \citet) son provistos por natbib
% No es necesario definirlos aquí

% -----------------------------------------------------------------------------
% VARIABLES GLOBALES DEL DOCUMENTO
% -----------------------------------------------------------------------------

% -----------------------------------------------------------------------------
% COMPONENTES DE CÓDIGOS DE IDENTIFICACIÓN
% -----------------------------------------------------------------------------
% Estos componentes se combinan automáticamente para generar los códigos completos

% Fecha en formato numérico dd/mm/yyyy (10 caracteres)
\newcommand{\fechaCodigo}{}
\newcommand{\setFechaCodigo}[1]{\renewcommand{\fechaCodigo}{#1}}

% Fecha en formato legible (para portada y documentación)
\newcommand{\fechaElaboracion}{}
\newcommand{\setFechaElaboracion}[1]{\renewcommand{\fechaElaboracion}{#1}}

% Código de la empresa (4 caracteres) - EKG: 1435
\newcommand{\codigoEmpresa}{}
\newcommand{\setCodigoEmpresa}[1]{\renewcommand{\codigoEmpresa}{#1}}

% Tipo de documento (2 caracteres) - NT: Nota Técnica, AN: Anexo
\newcommand{\tipoDocumento}{}
\newcommand{\setTipoDocumento}[1]{\renewcommand{\tipoDocumento}{#1}}

% Identificador del ramo (formato: letra-número) - Ejemplo: P-31
\newcommand{\ramo}{}
\newcommand{\setRamo}[1]{\renewcommand{\ramo}{#1}}

% Código interno del producto (16 caracteres alfanuméricos)
\newcommand{\codigoInterno}{}
\newcommand{\setCodigoInterno}[1]{\renewcommand{\codigoInterno}{#1}}

% Canal de comercialización (4 caracteres) - Ejemplo: D00I
\newcommand{\canalComercializacion}{}
\newcommand{\setCanalComercializacion}[1]{\renewcommand{\canalComercializacion}{#1}}

% Nombre del producto
\newcommand{\nombreProducto}{}
\newcommand{\setNombreProducto}[1]{\renewcommand{\nombreProducto}{#1}}

% -----------------------------------------------------------------------------
% CÓDIGOS COMPLETOS (GENERADOS AUTOMÁTICAMENTE)
% -----------------------------------------------------------------------------
% Estos se construyen automáticamente a partir de los componentes anteriores

% Código de Nota Técnica: fecha-empresa-tipo-ramo-interno
% Ejemplo: 03/10/2025-1435-NT-P-31-APINDMOVISTAR001
\newcommand{\codigoNotaTecnica}{\fechaCodigo-\codigoEmpresa-\tipoDocumento-\ramo-\codigoInterno}

% Código de Condicionado: fecha-empresa-ramo-interno-canal
% Ejemplo: 03/10/2025-1435-P-31-APINDMOVISTAR001-D00I
\newcommand{\codigoCondicionado}{\fechaCodigo-\codigoEmpresa-\ramo-\codigoInterno-\canalComercializacion}

\newcommand{\actuarioPrincipal}{}
\newcommand{\setActuarioPrincipal}[1]{\renewcommand{\actuarioPrincipal}{#1}}

\newcommand{\matriculaActuario}{}
\newcommand{\setMatriculaActuario}[1]{\renewcommand{\matriculaActuario}{#1}}

\newcommand{\credencialesActuario}{}
\newcommand{\setCredencialesActuario}[1]{\renewcommand{\credencialesActuario}{#1}}

% -----------------------------------------------------------------------------
% COMANDO PARA CREAR LA PORTADA
% -----------------------------------------------------------------------------
\newcommand{\crearPortada}{
    \begin{titlepage}
        \centering
        \vspace*{-2cm}
        
        % Título principal
        {\color{EKGAzulPrincipal}\titlefont\Huge\bfseries NOTA TÉCNICA}

        \vspace{1cm}

        % Nombre del producto
        {\color{EKGAzulPrincipal}\subtitlefont\LARGE\bfseries \nombreProducto}
        
        % Imagen de protección EKG ocupando toda la página de extremo a extremo
        \begin{tikzpicture}[remember picture,overlay]
            \node[inner sep=0pt] at ([yshift=1cm]current page.center) {%
                \includegraphics[width=\paperwidth,height=0.45\paperheight,keepaspectratio=false]{images/Protección EKG.jpg}%
            };
        \end{tikzpicture}

        \vspace{15.5cm} % Espacio para compensar la imagen y dar margen adicional
        
        % Elemento visual corporativo
        {\color{EKGAzulPrincipal}\rule{\textwidth}{3pt}}
        
        \vspace{0.8cm}
        
        % Tabla de información básica
        \begin{center}
        \begin{tabular}{|>{\columncolor{EKGAzulClaro}}p{6cm}|p{10cm}|}
        \hline
        Vigencia desde & \fechaElaboracion \\
        \hline
        Nota Técnica & \codigoNotaTecnica \\
        \hline
        Condicionado & \codigoCondicionado \\
        \hline
        \end{tabular}
        \end{center}
        
    \end{titlepage}
    \newpage
}

% -----------------------------------------------------------------------------
% COMANDO PARA TABLA DE TARIFAS ACTUARIALES
% -----------------------------------------------------------------------------
\newcommand{\tablaTarifas}[4]{
    \begin{table}[H]
        \centering
        \caption{#1}
        \begin{tabular}{|H|c|c|c|}
        \hline
        \rowcolor{EKGAzulPrincipal!20}
        \textbf{Edad} & \textbf{Hombres} & \textbf{Mujeres} & \textbf{Unisex} \\
        \hline
        #2 & #3 & #4 \\
        \hline
        \end{tabular}
        \label{tab:#1}
    \end{table}
}

% -----------------------------------------------------------------------------
% COMANDO PARA FÓRMULAS ACTUARIALES
% -----------------------------------------------------------------------------
\newcommand{\formulaActuarial}[3]{
    \begin{equation}
        \label{eq:#1}
        #2
    \end{equation}
    \begin{center}
        \textit{#3}
    \end{center}
}

% -----------------------------------------------------------------------------
% COMANDO PARA CREAR GRÁFICO ACTUARIAL
% -----------------------------------------------------------------------------
\newcommand{\graficoActuarial}[4]{
    \begin{figure}[H]
        \centering
        \begin{tikzpicture}
            \begin{axis}[
                ekg style,
                xlabel={#2},
                ylabel={#3},
                title={#1}
            ]
            #4
            \end{axis}
        \end{tikzpicture}
        \caption{#1}
        \label{fig:#1}
    \end{figure}
}

% -----------------------------------------------------------------------------
% COMANDO PARA DESTACAR INFORMACIÓN REGULATORIA
% -----------------------------------------------------------------------------
\newcommand{\infoRegulatoria}[1]{
    \begin{center}
    \fcolorbox{EKGAzulPrincipal}{EKGAzulClaro}{
        \begin{minipage}{0.9\textwidth}
            \textcolor{EKGGrisOscuro}{\textbf{INFORMACIÓN REGULATORIA:}}
            \vspace{0.3cm}
            
            #1
        \end{minipage}
    }
    \end{center}
}

% -----------------------------------------------------------------------------
% COMANDOS MATEMÁTICOS ACTUARIALES
% -----------------------------------------------------------------------------

% Comando para esperanza matemática
\newcommand{\E}{\mathbb{E}}

% Comando para probabilidad
\newcommand{\Prob}{\mathbb{P}}

% Comando para varianza
\newcommand{\Var}{\text{Var}}

% -----------------------------------------------------------------------------
% COMANDO PARA CÓDIGO DE NOTA TÉCNICA EN FORMATO REQUERIDO
% -----------------------------------------------------------------------------
\newcommand{\mostrarCodigoCompleto}{
    \begin{center}
    \textcolor{EKGAzulPrincipal}{\textbf{Código de Identificación:}} \\
    \texttt{\Large \codigoNotaTecnica}
    \end{center}
    
    \textbf{Estructura del código:}
    \begin{itemize}
        \item \textbf{Fecha de utilización:} Formato dd/mm/aaaa
        \item \textbf{Tipo de entidad:} 1435 (EKG - Aseguradora de Vida)
        \item \textbf{Tipo de documento:} NT-P (Nota Técnica de Producto)
        \item \textbf{Ramo:} Según clasificación Superfinanciera
        \item \textbf{Identificación interna:} Código alfanumérico único
    \end{itemize}
}

% -----------------------------------------------------------------------------
% ENTORNO PARA BASES TÉCNICAS
% -----------------------------------------------------------------------------
\newenvironment{basestecnicas}
{
    \section{Bases Técnicas}
    \color{EKGGrisOscuro}
}
{
    \color{black}
}

% -----------------------------------------------------------------------------
% COMANDO PARA CREAR ANEXOS
% -----------------------------------------------------------------------------
\newcommand{\crearAnexo}[2]{
    \section{#1}
    \label{anexo:#1}
    #2
}

% -----------------------------------------------------------------------------
% CONFIGURACIÓN PARA BLOQUES DE CÓDIGO PYTHON
% -----------------------------------------------------------------------------

% Definir fuente Mononoki Nerd Font para código
% XeLaTeX busca la fuente automáticamente en las ubicaciones estándar del sistema
\newfontfamily\mononokifont{Mononoki Nerd Font}[
    BoldFont=Mononoki Nerd Font Bold,
    ItalicFont=Mononoki Nerd Font Italic,
    BoldItalicFont=Mononoki Nerd Font Bold Italic,
    Scale=0.9
]

% Paleta de colores Gitpod Light Theme
\definecolor{GitpodBackground}{RGB}{255, 255, 255}    % Fondo blanco
\definecolor{GitpodForeground}{RGB}{0, 0, 0}          % Texto principal negro
\definecolor{GitpodComment}{RGB}{106, 115, 125}       % Comentarios grises
\definecolor{GitpodKeyword}{RGB}{0, 92, 197}          % Keywords azul
\definecolor{GitpodString}{RGB}{10, 136, 69}          % Strings verde
\definecolor{GitpodNumber}{RGB}{9, 134, 88}           % Números verde azulado
\definecolor{GitpodFunction}{RGB}{121, 93, 163}       % Funciones púrpura
\definecolor{GitpodConstant}{RGB}{0, 92, 197}         % Constantes azul
\definecolor{GitpodModule}{RGB}{4, 81, 165}           % Módulos azul oscuro
\definecolor{GitpodFrame}{RGB}{200, 200, 200}         % Marco gris claro

\lstdefinestyle{pythonStyle}{
    language=Python,
    basicstyle=\small\mononokifont\color{GitpodForeground},
    keywordstyle=\color{GitpodKeyword}\bfseries,
    commentstyle=\color{GitpodComment}\itshape,
    stringstyle=\color{GitpodString},
    numberstyle=\tiny\color{GitpodComment},
    identifierstyle=\color{GitpodForeground},
    emph={self,True,False,None},
    emphstyle=\color{GitpodConstant}\bfseries,
    emph={[2]pd,np,DataFrame},
    emphstyle={[2]\color{GitpodModule}\bfseries},
    emph={[3]def,class},
    emphstyle={[3]\color{GitpodKeyword}\bfseries},
    numbers=left,
    numbersep=10pt,
    stepnumber=1,
    backgroundcolor=\color{GitpodBackground},
    frame=single,
    framesep=8pt,
    frameround=tttt,
    rulecolor=\color{GitpodFrame},
    breaklines=true,
    breakatwhitespace=true,
    showstringspaces=false,
    tabsize=4,
    captionpos=b,
    xleftmargin=20pt,
    xrightmargin=10pt,
    aboveskip=15pt,
    belowskip=15pt,
    columns=flexible,
    keepspaces=true,
}

% Estilo para output de consola/terminal
\definecolor{OutputBackground}{RGB}{240, 244, 248}    % Fondo gris azulado claro
\definecolor{OutputText}{RGB}{36, 41, 46}             % Texto gris oscuro
\definecolor{OutputFrame}{RGB}{149, 157, 165}         % Marco gris medio

\lstdefinestyle{outputStyle}{
    language={},
    basicstyle=\small\mononokifont\color{OutputText},
    numbers=none,
    backgroundcolor=\color{OutputBackground},
    frame=single,
    framesep=8pt,
    frameround=tttt,
    rulecolor=\color{OutputFrame},
    breaklines=true,
    breakatwhitespace=false,
    showstringspaces=false,
    tabsize=4,
    captionpos=b,
    xleftmargin=20pt,
    xrightmargin=10pt,
    aboveskip=15pt,
    belowskip=15pt,
    columns=flexible,
    keepspaces=true,
}

% Comando para insertar código Python
\newcommand{\codigoPython}[2][]{
    \lstinputlisting[style=pythonStyle, caption={#1}]{#2}
}

% Comando para código Python inline
\lstnewenvironment{pythoncode}[1][]
{
    \lstset{style=pythonStyle, #1}
}
{}