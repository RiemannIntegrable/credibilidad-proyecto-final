% =============================================================================
% ESTILOS CORPORATIVOS EKG - IDENTIDAD VISUAL PARA NOTA TÉCNICA (SIMPLIFICADO)
% =============================================================================

% -----------------------------------------------------------------------------
% CONFIGURACIÓN DE COLORES CORPORATIVOS EKG
% -----------------------------------------------------------------------------
% Según Manual de Marca EKG 2024 - Paletas de Color (pág. 10-11)

% Color Palette 01 - Primaria #5B677B
\definecolor{EKGAzulPrincipal}{HTML}{5B677B}
% Color Palette 02 - Primaria #979797
\definecolor{EKGGris}{HTML}{979797}
% Color Palette 03 - Secundaria #769BC5 (máximo 20% de proporción)
\definecolor{EKGAzulSecundario}{HTML}{769BC5}

% Colores adicionales para elementos del documento
\definecolor{EKGAzulClaro}{HTML}{E8EDEF}
\definecolor{EKGGrisOscuro}{HTML}{4A4A4A}
% Texto normal en negro puro para máxima legibilidad
\definecolor{TextoNormal}{HTML}{000000}

% -----------------------------------------------------------------------------
% CONFIGURACIÓN DE TIPOGRAFÍA CORPORATIVA OFICIAL EKG
% -----------------------------------------------------------------------------
% Según Manual de Marca EKG 2024 - Tipografías Primarias /Uso corporativo (pág. 13)

% Fuente principal para texto normal - Microsoft Sans Serif OFICIAL
% Uso: Párrafo texto libre. Color principal gris, secundario azul
% XeLaTeX busca la fuente automáticamente en las ubicaciones estándar del sistema
\setmainfont{Microsoft Sans Serif}[
    BoldFont=Microsoft Sans Serif,
    BoldFeatures={FakeBold=2.2}
]
\setsansfont{Microsoft Sans Serif}[
    BoldFont=Microsoft Sans Serif,
    BoldFeatures={FakeBold=2.2}
]
\setmonofont{DejaVu Sans Mono}[Scale=0.95]

% Definir fuentes para jerarquía de títulos según Manual EKG

% FUENTES CORPORATIVAS OFICIALES según Manual EKG (pág. 13):
% - Aharoni: Uso: Títulos de secciones o capítulos (Color principal azul, secundario gris)
% - Century Gothic: Uso: Títulos y subtítulos principalmente (Color principal azul, secundario gris)

% Fuente para títulos de secciones o capítulos - Aharoni OFICIAL
% Uso: Títulos de secciones o capítulos. Color principal azul, secundario gris
% XeLaTeX busca la fuente automáticamente en las ubicaciones estándar del sistema
\newfontfamily\titlefont{Aharoni}[
    BoldFont=Aharoni,
    Scale=1.0,
    Ligatures=TeX
]

% Fuente para títulos y subtítulos - Century Gothic OFICIAL
% Uso: Títulos y subtítulos principalmente. Color principal azul, secundario gris
% XeLaTeX busca la fuente automáticamente en las ubicaciones estándar del sistema
\newfontfamily\subtitlefont{Century Gothic}[
    BoldFont=Century Gothic Bold,
    ItalicFont=Century Gothic Italic,
    BoldItalicFont=Century Gothic Bold Italic,
    Ligatures=TeX
]

% Configuración completa: todas las fuentes corporativas están disponibles
% Microsoft Sans Serif para texto normal, Aharoni para títulos principales,
% Century Gothic para subtítulos según Manual de Marca EKG 2024

% -----------------------------------------------------------------------------
% CONFIGURACIÓN DE PÁGINA Y GEOMETRÍA
% -----------------------------------------------------------------------------
\geometry{
    left=2cm,
    right=2cm,
    top=3.5cm,
    bottom=2cm,
    headheight=66pt,
    headsep=0.8cm,
    footskip=1cm
}

% -----------------------------------------------------------------------------
% LOGO VIGILADO SFC EN ESQUINA SUPERIOR IZQUIERDA (TODAS LAS PÁGINAS)
% -----------------------------------------------------------------------------
% Colocar logo "Vigilado SFC" en la esquina superior izquierda absoluta
% de todas las páginas (incluida portada) con opacidad reducida
\AddToShipoutPictureBG{%
    \begin{tikzpicture}[remember picture,overlay]
        \node[opacity=0.3, inner sep=0pt] at (current page.north west) [anchor=north west] {%
            \includegraphics[width=1.2cm]{images/vigilado_sfc.png}%
        };
    \end{tikzpicture}%
}

% -----------------------------------------------------------------------------
% CONFIGURACIÓN DE ENCABEZADOS Y PIES DE PÁGINA
% -----------------------------------------------------------------------------
\pagestyle{fancy}
\fancyhf{}

% Redefinir marcas de sección para NO usar mayúsculas automáticamente
\renewcommand{\sectionmark}[1]{\markboth{#1}{}}

% Encabezado con textos alineados a la izquierda y logo a la derecha
\fancyhead[L]{
    \begin{tabular}[b]{@{}l@{}}
    \textcolor{EKGAzulPrincipal}{\small\titlefont\textbf{Nota Técnica}} \\
    \textcolor{EKGAzulPrincipal}{\scriptsize\subtitlefont\nombreProducto} \\
    \textcolor{EKGGris}{\scriptsize\subtitlefont\leftmark}
    \end{tabular}
}
\fancyhead[R]{
    \includegraphics[height=1.5cm]{images/losactuapitwo2.png}
}

% Pie de página
\fancyfoot[L]{
    \begin{tabular}[t]{@{}l@{}}
    \tiny\textcolor{EKGGris}{Nota técnica: \codigoNotaTecnica} \\
    \tiny\textcolor{EKGGris}{Condicionado: \codigoCondicionado}
    \end{tabular}
}
\fancyfoot[C]{}
\fancyfoot[R]{\footnotesize\textcolor{EKGGris}{\thepage}}

% Líneas de encabezado y pie
\renewcommand{\headrulewidth}{2pt}
\renewcommand{\headrule}{\hbox to\headwidth{\color{EKGAzulPrincipal}\leaders\hrule height \headrulewidth\hfill}}
\renewcommand{\footrulewidth}{0.5pt}
\renewcommand{\footrule}{\hbox to\headwidth{\color{EKGGris}\leaders\hrule height \footrulewidth\hfill}}

% -----------------------------------------------------------------------------
% CONFIGURACIÓN DE TÍTULOS Y SECCIONES
% -----------------------------------------------------------------------------
% Implementación de jerarquía tipográfica según Manual EKG

% Formato de secciones (títulos de secciones/capítulos - Aharoni)
% Color principal azul según Manual EKG
% NOTA: Por defecto centrado para índices. Usar \aplicarFormatoSeccionesContenido para alinear a la derecha
% Tamaño: \Large (17.28pt) para texto, \Huge (24.88pt) para número
\titleformat{\section}
    {\titlefont\Large\bfseries\color{EKGAzulPrincipal}}
    {\titlefont\Huge\bfseries\thesection}{1em}{}

% Formato de subsecciones (títulos y subtítulos - Century Gothic)
% Color principal azul, secundario gris según Manual EKG
% Tamaño: \large (14.4pt) para segundo nivel de jerarquía
\titleformat{\subsection}
    {\subtitlefont\large\bfseries\color{EKGAzulPrincipal}}
    {\thesubsection}{1em}{}

% Formato de subsubsecciones (Century Gothic)
% Color secundario gris para jerarquía visual
% Tamaño: \large (14.4pt) para tercer nivel de jerarquía - más grande que texto normal
\titleformat{\subsubsection}
    {\subtitlefont\large\bfseries\color{EKGGrisOscuro}}
    {\thesubsubsection}{1em}{}

% Formato de párrafos numerados (paragraph)
% Usar para divisiones menores dentro de subsubsecciones
% Tamaño: \normalsize (12pt, mismo que texto normal) con color más claro
\titleformat{\paragraph}
    {\normalfont\normalsize\bfseries\color{EKGGris}}
    {\theparagraph}{1em}{}

% Formato de subpárrafos (subparagraph)
% Tamaño: \small (10.95pt) para mínima jerarquía visual
\titleformat{\subparagraph}
    {\normalfont\small\bfseries\color{EKGGrisOscuro}}
    {\thesubparagraph}{1em}{}

% -----------------------------------------------------------------------------
% COMANDO PARA ACTIVAR FORMATO DE SECCIONES DEL CONTENIDO PRINCIPAL
% -----------------------------------------------------------------------------
% Este comando redefine el formato para diferenciar jerárquicamente cada nivel
% Llamar justo antes de iniciar el contenido principal (después de índices)
\newcommand{\aplicarFormatoSeccionesContenido}{
    % Secciones: CENTRADAS con número grande
    \titleformat{\section}
        {\titlefont\Large\bfseries\color{EKGAzulPrincipal}\centering}
        {\titlefont\Huge\bfseries\thesection}{1em}{}

    % Subsecciones: ALINEADAS A LA DERECHA
    \titleformat{\subsection}
        {\subtitlefont\large\bfseries\color{EKGAzulPrincipal}\raggedleft}
        {\thesubsection}{1em}{}

    % Subsubsecciones: ALINEADAS A LA IZQUIERDA - tamaño más grande para diferenciarse
    \titleformat{\subsubsection}
        {\subtitlefont\large\bfseries\color{EKGGrisOscuro}}
        {\thesubsubsection}{1em}{}

    % Párrafos: ALINEADOS A LA IZQUIERDA - tamaño normal, color más claro
    \titleformat{\paragraph}
        {\normalfont\normalsize\bfseries\color{EKGGris}}
        {\theparagraph}{1em}{}
}

% -----------------------------------------------------------------------------
% CONFIGURACIÓN DE TABLAS
% -----------------------------------------------------------------------------

% Estilo para encabezados de tabla
\newcolumntype{H}{>{\columncolor{EKGAzulPrincipal!20}\bfseries\color{EKGGrisOscuro}}c}
\newcolumntype{C}{>{\centering\arraybackslash}X}

% -----------------------------------------------------------------------------
% CONFIGURACIÓN DE LISTAS
% -----------------------------------------------------------------------------

% Configuración de enumeración
\setlist[enumerate,1]{
    label=\textbf{\arabic*.},
    leftmargin=*,
    itemsep=0.5em,
    topsep=0.5em
}

% Configuración de viñetas
\setlist[itemize,1]{
    label=\textcolor{EKGAzulPrincipal}{\textbullet},
    leftmargin=*,
    itemsep=0.3em,
    topsep=0.3em
}

% -----------------------------------------------------------------------------
% CONFIGURACIÓN DE HIPERVÍNCULOS BÁSICA
% -----------------------------------------------------------------------------
\hypersetup{
    colorlinks=true,
    linkcolor=EKGAzulPrincipal,
    filecolor=EKGAzulPrincipal,
    urlcolor=EKGAzulSecundario,
    citecolor=EKGGrisOscuro,
    bookmarksnumbered=true,
    bookmarksopen=true
}

% Comando para configurar metadata PDF después de definir las variables
\newcommand{\configurarMetadataPDF}{
    \hypersetup{
        pdftitle={Nota Técnica - \nombreProducto},
        pdfauthor={\actuarioPrincipal - EKG Compañía de Seguros de Vida S.A.},
        pdfsubject={Nota Técnica Actuarial - Seguros de Vida Individual},
        pdfkeywords={seguros, vida individual, nota técnica, actuarial, EKG, Colombia},
        pdfcreator={XeLaTeX con plantilla EKG},
        pdfproducer={EKG Compañía de Seguros de Vida S.A.}
    }
}

% -----------------------------------------------------------------------------
% CONFIGURACIÓN DE PGFPLOTS PARA GRÁFICOS
% -----------------------------------------------------------------------------
\pgfplotsset{
    /pgfplots/ekg style/.style={
        width=0.8\textwidth,
        height=0.6\textwidth,
        grid=major,
        grid style={line width=0.1pt, draw=EKGGris!30},
        axis lines*=left,
        xlabel style={color=EKGGrisOscuro, font=\sffamily},
        ylabel style={color=EKGGrisOscuro, font=\sffamily},
        title style={color=EKGAzulPrincipal, font=\sffamily\bfseries},
        legend style={
            at={(0.02,0.98)},
            anchor=north west,
            font=\footnotesize\sffamily,
            fill=white,
            fill opacity=0.9,
            draw=EKGGris
        },
        every axis plot/.append style={
            line width=1.5pt,
            color=EKGAzulPrincipal
        }
    }
}

% -----------------------------------------------------------------------------
% ESPACIADO Y CONFIGURACIÓN GENERAL DEL DOCUMENTO
% -----------------------------------------------------------------------------
\setstretch{1.15}
\setlength{\parindent}{0pt}
\setlength{\parskip}{0.8em}

% Configurar color de texto normal en negro puro
\AtBeginDocument{\color{TextoNormal}}

% -----------------------------------------------------------------------------
% CONFIGURACIÓN DEL ÍNDICE
% -----------------------------------------------------------------------------
\setcounter{tocdepth}{4} % Mostrar hasta paragraph (nivel 4) - Con jerarquía completa
\setcounter{secnumdepth}{4} % Numerar hasta paragraph (nivel 4)

% Cambiar títulos de índices a Title Case (sin mayúsculas obligadas)
\addto\captionsspanish{%
  \renewcommand{\contentsname}{Índice}%
  \renewcommand{\listtablename}{Índice de Tablas}%
  \renewcommand{\listfigurename}{Índice de Figuras}%
}

% Configuración mínima del TOC: solo fuente más pequeña para párrafos
\titlecontents{paragraph}
    [10em]                          % Sangría por defecto
    {\small}                        % Fuente más pequeña para párrafos
    {\contentslabel{4em}}           % Espacio para número
    {}
    {\titlerule*[0.5pc]{.}\contentspage}


% -----------------------------------------------------------------------------
% CONFIGURACIÓN DE NUMERACIÓN ROMANA PARA SECCIONES PRELIMINARES
% -----------------------------------------------------------------------------
% Redefinir formato de numeración de secciones para usar romanos
\newcommand{\usarNumeracionRomana}{
    \renewcommand{\thesection}{\Roman{section}}
    \renewcommand{\thesubsection}{\Roman{section}.\roman{subsection}}
}

% Redefinir formato de numeración de secciones para usar arábigos
\newcommand{\usarNumeracionArabiga}{
    \renewcommand{\thesection}{\arabic{section}}
    \renewcommand{\thesubsection}{\arabic{section}.\arabic{subsection}}
    \renewcommand{\thesubsubsection}{\arabic{section}.\arabic{subsection}.\arabic{subsubsection}}
    \renewcommand{\theparagraph}{\arabic{section}.\arabic{subsection}.\arabic{subsubsection}.\arabic{paragraph}}
    \renewcommand{\thesubparagraph}{\arabic{section}.\arabic{subsection}.\arabic{subsubsection}.\arabic{paragraph}.\arabic{subparagraph}}
}

% -----------------------------------------------------------------------------
% CONFIGURACIÓN DE APÉNDICES
% -----------------------------------------------------------------------------
\renewcommand{\appendixname}{Anexo}
\renewcommand{\appendixtocname}{Anexos}
\renewcommand{\appendixpagename}{Anexos}

% -----------------------------------------------------------------------------
% ESTILOS PERSONALIZADOS PARA DIFERENTES SECCIONES
% -----------------------------------------------------------------------------

% Estilo para contenido principal (con numeración página/total)
\fancypagestyle{contenidoprincipal}{
    \fancyhf{}
    \fancyhead[L]{
        \begin{tabular}[b]{@{}l@{}}
        \textcolor{EKGAzulPrincipal}{\small\titlefont\textbf{Nota Técnica}} \\
        \textcolor{EKGAzulPrincipal}{\scriptsize\subtitlefont\nombreProducto} \\
        \textcolor{EKGGris}{\scriptsize\subtitlefont\leftmark}
        \end{tabular}
    }
    \fancyhead[R]{\includegraphics[height=1.5cm]{images/losactuapitwo2.png}}
    \fancyfoot[L]{
        \begin{tabular}[t]{@{}l@{}}
        \tiny\textcolor{EKGGris}{Nota técnica: \codigoNotaTecnica} \\
        \tiny\textcolor{EKGGris}{Condicionado: \codigoCondicionado}
        \end{tabular}
    }
    \fancyfoot[C]{}
    \fancyfoot[R]{\footnotesize\textcolor{EKGGris}{\thepage/\pageref{LastPage}}}
    \renewcommand{\headrulewidth}{2pt}
    \renewcommand{\headrule}{\hbox to\headwidth{\color{EKGAzulPrincipal}\leaders\hrule height \headrulewidth\hfill}}
    \renewcommand{\footrulewidth}{0.5pt}
    \renewcommand{\footrule}{\hbox to\headwidth{\color{EKGGris}\leaders\hrule height \footrulewidth\hfill}}
}

% Estilo para portada (sin códigos en pie)
\fancypagestyle{portada}{
    \fancyhf{}
    \renewcommand{\headrulewidth}{0pt}
    \renewcommand{\footrulewidth}{0pt}
}